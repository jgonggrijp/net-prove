\documentclass{article}
\usepackage[a4paper]{geometry}
\usepackage{amssymb} 
\usepackage{amsmath} 
\usepackage{stmaryrd}
\usepackage{graphicx}
\usepackage{latexsym}
\usepackage{proof}
\usepackage{mathabx}
\usepackage{xcolor}
\usepackage[british]{babel}

\usepackage{fontspec,xltxtra,xunicode}
\defaultfontfeatures{Mapping=tex-text}
\setromanfont[]{Palatino}
\setsansfont[Scale=MatchLowercase]{Helvetica}
\setmonofont[Scale=MatchLowercase]{FreeMono}

\usepackage{minted}

\newcommand{\nd}[2]{#1 \vdash #2}
\newcommand{\Ra}{\rightarrow}
\newcommand{\lolli}{\multimap}
\newcommand{\bs}{\backslash}
\newcommand{\W}[1]{\textsf{#1}}
\newcommand{\ld}{\lambda}

\begin{document}

\title{Term derivation in automated proof nets for the Lambek-Grishin calculus}
\author{Julian Gonggrijp\thanks{Joint project with Niels Steenbergen and Maarten Trompper}}
\date{}
\maketitle

\section{Introduction}

The Lambek-Grishin calculus (LG) is an extension of Lambek calculus that has proven itself in the analysis of several context-sensitive natural language phenomena. sLG, a display sequent calculus for LG, has been shown to be tractable. \cite{m09} However, it allows for spurious ambiguity, requires explicit manipulation of sequents that are equivalent under structural rules and depends on a predetermined sequent structure. For the latter reason, sLG alone cannot be used to parse a sentence. Moortgat and Moot (2012) provided a proof net formalism for LG that solves all of these issues. \cite{mm12}

Automated theorem provers tend to work in ``sequent style'', searching top-down by backwards chaining. A theorem prover based on proofnets could work bottom-up instead and compute the proof structure---hence, parse a sentence---as a byproduct. To our knowledge, such a theorem prover was not yet built for LG. We decided to fill this gap using Haskell. The context for this endeavour was the 2014--2015 Master's course \emph{Logic and Language}, taught by Prof. dr. Michael Moortgat at Utrecht University\footnote{link}.

After defining a core datastructure as the interface between components of the theorem prover, we devided the remaining work into three subprojects that could be developed in parallel. Term derivation, the topic of the current paper, is one of those three subprojects. By the Curry-Howard isomorphism, every proof has a compact, algebraic representation called a term; since a proof net is also a proof, it follows that it can be represented with a term. The task of the author was to realise this for our Haskell implementation of LG proof nets. The term would then be available as a universal interface for further use; as a convenient proof representation to the human reader, as a source from which to reconstruct the display sequent version of the proof, or as the input material to a homomorphic mapping to another calculus in a categorial grammar setup. In particular, a CPS translation to MILL/LP has been shown to be worthwhile when LG is the source calculus. \cite{mm12}

\inputminted{haskell}{../src/LG/Term.hs}

\inputminted{haskell}{../src/LG/Subnet.hs}

\inputminted{haskell}{../src/LG/SubnetGraph.hs}

\inputminted{haskell}{../src/LG/TestGraph.hs}

\end{document}
